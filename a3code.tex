\documentclass[10pt,a4paper,final]{article}
\usepackage[utf8]{inputenc}
\usepackage[german]{babel}
\usepackage{amsmath}
\usepackage{amsfonts}
\usepackage{amssymb}
\usepackage{makeidx}
\usepackage{graphicx}
\usepackage[left=2cm,right=2cm,top=2cm,bottom=2cm]{geometry}
\usepackage{color}
\author{Leandro Marcelo Torres}
\title{A3-Code}
\date{\today}

\begin{document}
\maketitle %make a comment
\thispagestyle{empty}
\cleardoublepage
\tableofcontents
\newpage
\section{Allgemeines}
\subsection{Genehmigung, Modifizierung und Übersetzung des Codes}
\subsubsection{Genehmigung}
Der A3-Code wird zum ersten Mal an der Flurversammlung genehmigt. Er muss von mindestens der Hälfte der ordentlichen Mitbewohnern genehmigt werden. Wurde der A3-Code von einem Mitbewohner nicht gelesen, so zählt die Stimme dieses Mitbewohners automatisch zur Genehmigung des A3-Codes. 
\subsubsection{Modifizierung}
Der A3-Code kann nur auf Wunsch der Flurversammlung geändert werden. Eine Änderung des A3-Codes ohne die Genehmigung der Flurversammlung ist nicht möglich.\\
Falls eine Änderung des Codes seitens eines Mitbewohners erwünscht ist, so muss sich dieser Mitbewohner darum kümmern, diesen Änderungswunsch an der Flurversammlung bekannt zu machen.
\subsubsection{Übersetzung des Codes}
Für alle nicht deutschsprachige Mitbewohner wird eine englische Version des A3-Codes zur Verfügung gestellt.\\
\section{Neueinzügler}
Jede Person, die neu in A3 einzieht wird in folgenden als Neueinzügler definiert.\\
Neueinzügler sind i.a. Personen, die nicht vorher in einem Wohnheim gewohnt haben.\\
Sie sind dementsprechend eine potenzielle Gefährdung für die Fluranlagen, da sie die Regel vom Flur nicht kennen.\\

\subsection{Neueinzüglersunterweisung}
Wer zuerst einen Neueinzügler sieht, muss ihn darauf hinweisen, dass es einen A3-Code gibt, und mit ihm die wichtigsten Punkte von dieses Codes durchgehen. Sollte diese Person keine Zeit dafür haben, muss sie eine andere Person suchen, die dem Neueinzügler alles erklärt.\\
Besonders von Interesse sind die Punkten, bei deren nicht Beachtung etwas kaputt gehen kann (Teflonpfannen, Couchs, usw.) oder andere Mitbewohner gestört werden können (unangemeldete Partys, nicht Erledigung des Mülldiensts, Lautstärke im Flur).
\subsection{Neueinzüglersessen}
Eine sehr alte Tradition vom A3-Code ist das Neueinzüglersessen.\\
Hier kümmern sich die Neueinzügler, ein Essen für den ganzen Flur vorzubereiten. Die Kosten werden durch die Flurkasse übernommen.
\section{Sauberkeit}
\subsection{Flurputz}
\subsubsection{Allgemeines}
Der Flurputz findet einmal im Semester statt. Alle Mitbewohner sind verpflichtet, daran teilzunehmen. Der Termin wird an der Flurversammlung festgelegt. Der Termin wird so festelegt, dass die meisten Mitbewohner daran teilnehmen können. Sollte ein Mitbewohner aus einem wichtigen Grund nicht teilnehmen können, so bekommt er Aufgaben zugeteilt, deren Erfüllung nicht die Anwesenheit des Mitbewohners am Flurputz erfordern. Die Teilnehmer an der Flurversammmlung entscheiden über die Gültigkeit der des Mitbewohners angegebenen Gründe, weswegen er am Flurputz nicht anwesend sein kann. 
Das Putzkriterium ist immer von Oben nach unten (Arbeitsflächen werden vor dem Boden sauber gemacht).
Alle Gegenstände, die bewegbar sind, werden auf den Gang gelegt und dort saubergemacht (Nicht im Teppichbereich, sondern im Normalbodenbereich)
Die Klos werden zuletzt sauber gemacht, da das dreckige Wasser von allen Reinigungsvorgängen in die Klos geschüttet wird, und da somit die Toiletten über die ganze Dauer des Flurputzes nutzbar sind.
\subsubsection{Zu reinigende Bereiche und Gegenstände}
\paragraph{Küche:}
\begin{itemize}
\item Fenster
\item Boden (Mülltonnenbereich insbesondere)
\item Arbeitsflächen
\item Herdplatten
\item Öfen
\item Waschbecken und Abtropfbereich
\item Getränkenkühlschrank
\item Gefrierschrank
\item Bewegliche Gegenstände
\end{itemize}

\paragraph{Wohnzimmer:}
\begin{itemize}
\item Fenster
\item Boden
\item Heizung
\item Couchs
\item Bewegliche Gegenstände

\end{itemize}

\paragraph{Gang:}
\begin{itemize}
\item Teppich (saugen)
\item Müll sammeln

\end{itemize}
\paragraph{Balkon:}
\begin{itemize}
\item Müll sammeln
\item Grill saubermachen (der Grillrost wird nicht gespült)
\item Biergarnitur abwischen (nicht mit Wasser)
\item Boden wischen
\item Bewegliche Gegenstände
\item Glastür
\end{itemize}

\paragraph{Bad:} 
\begin{itemize}
\item Klos mit Toilettenbürste sauber machen
\item Duschen (obere Kanten der Duschentüre nicht vergessen) und Abläufe reinigen
\item Müll hinaus bringen
\item Waschmaschine und Trockner hinaus tragen und deren Abstellbereiche reinigen
\item Tür und Fenster
\item Heizung
\end{itemize}
\subsubsection{Erforderliche Reinigungsmittel}
\begin{itemize}
\item Glassreinigungsmittel
\item Bodenreinigungsmittel
\item Ofenreinigungsmittel
\item Holzreinigungsmittel
\item Entkalker
\item Pilzenentfernungsmittel (Chlor o.ä.)
\item Klosreinigungsmittel
\item 7 Paar Handschuhe
\item Flächenreiniger (für die Küchenarbeitsflächen)
\item 3 Bodenlappen
\item Lappen zum Fenster reinigen (Orangen Lappen)
\end{itemize}

\section{Gäste}
\subsection{Veranstaltungsankündigung} Die Einladung von mehr als 4 Gästen muss 72 Stunden im Vorfeld angekündigt werden. Die Ankündigung erfolgt per E-mail oder schriftlich (White Board) auf Deutsch oder Englisch. Falls die Ankündigung nicht rechtzeitig erfolgt hat jeder Mitbewohner, der sich im Wohnzimmer-bzw. Küchenbereich aufhalten möchte das Recht die Gäste bzw. Den Gastgeber darum zu bitten, die Räume zu verlassen. (siehe Kritikenparagraph).
\subsection{Haftung}  Jeder Gastgeber haftet für das Verhalten seiner Gäste. (Stoßen gegen den A3-Code).
\subsection{Beachtung der Privatsphäre} Falls die Veranstaltung rechtzeitig angekündigt wurde, so haben andere Mitbewohner die Privatsphäre der Veranstaltung zu beachten. Dies bedeutet, dass sich keiner Mitbewohner zu einer Veranstaltung eines anderen Mitbewohners selber einladen kann oder in ihr Gespräch einmischen kann, es sei denn, er ist auch ein Gast. Falls solch ein Ereignis vorkommt, kann der Gastgeber den Mitbewohner darauf hinweisen (Siehe Kritikenparagraph).
\subsection{Freundlichkeit} Alle Mitbewohner haben sich mit den Gästen von anderen Mitbewohnern freundlich zu verhalten. Die Unfreundlichtkeit Gästen anderer Mitbewohner gegenüber stellt ein Stoß gegen den A3-Code dar.

\section{Küche}
\subsection{Teflonpfannen}
Die Teflonpfannen dürfen nur mit Holz-oder Kunststoffwerkzeuge benutzt werden. Dies bedeutet, dass man beim Kochen, Essen oder Servieren keine metallische Gegenstände in der Hand haben kann.\\
Als praktische Regel ist es nicht erlaubt, eine Teflonpfanne in einer Hand zu haben und gleichzeitig etwas metallisches in der anderen Hand zu haben.\\
Die Teflonpfannen dürfen aus dem selben Grund nur mit der gelben Seite des Schwamms gespült werden.\\
Nach dem Spülen dürfen die Teflonpfannen nicht in den Abtropfgitter gelegt werden. Sie müssen langsam auf dem Bereich rechts vom Waschbecken außerhalb des Abtropfgitters abgelegt und als erste abgetrocknet werden.\\
Gewisse Teflonpfannen haben eine kleine Marke, die erst Rot wird wenn die Pfanne heiß ist. Um solche Pfannen zu spülen muss man erstmal warten, dass diese Marke nicht mehr rot ist. Die Pfannen dürfen keinerlei im Kontakt mit Wasser kommen, wenn die Marke rot ist. 
\subsection{Kochen-und Essensablauf}
\subsubsection{Kochen} Beim Kochen nicht die ganze Arbeitsfläche benutzen, es muss auch Platz für jemanden anderen geben, der auch kochen möchte.
\subsubsection{Arbeitsfläche Abwischen} Das muss sofort nach dem Kochen geschehen. Sonst muss es der nächste machen, während Du isst.
\subsubsection{Herdplattenbereich frei lassen} Siehe Grund vom Punkt 2
\subsubsection{Essen}
\subsubsection{Tisch und Boden abwischen} Siehe Grund vom Punkt 2.\\ Der Tisch wird mit dem Lappen abgewischt. Der Boden wird mit einem Bodenlappen abgewischt, oder mit dafür vorgesehenen Schwamm (Dieser Schwamm ist der älteste und dreckigste von allen und befindet sich wo die Mülltüten sind)
\subsubsection{Geschirr spülen}
\subsubsection{Herdplattenbereich sauber machen} Der Herd muss mit dem dafür vorgesehener Schwamm (am schlechtesten aussehenden Schwamm) abgewischt werden. Das Abwischen vom Herd erfolgt mit Spülmittel. Die Ränder vom Herd sowie die Arbeitsfläche muss auch sauber gemacht werden (Besonders beim Braten, wenn das ganze mit Öl verspritz wurde)
\subsubsection{Geschirr abtrocknen und wegräumen} (Siehe Geschirr)


\subsection{Geschirr}
\subsubsection{Allgemeines}
Das Geschirr muss nach dem Essen sofort gespült werden. Die hier enthaltenen Paragraphen sind lediglich die Tolereranzgrenzen, die der A3-Code akzeptieren wird. Nur Teller, Besteckt, Tassen und Gläser dürfen begrenzt in die Zimmer mitgenommen werden. Alle andere Kochmittel-und Werkzeuge dürfen die Küche nicht verlassen. Das Mitnehmen aller anderen Kochmittel ins Zimmer stellt einen Verstoß gegen den A3-Code dar.
\subsubsection{Vorgehen}
Das Geschirr muss stets mit Schwamm und Spülmittel gespült werden. Nach dem Spülen jedes Geschirrteils muss dieses augenscheinlich nach Schmutz, Öl, und/oder Fett überprüft werden.
\subsubsection{Zeitgrenzen}
Das Geschirr darf i.A. 12h oder bis es dunkel wird auf der grünen Fläche stehen. (Das gilt für Winter und Sommer). Diese Regel gilt nur für das Geschirr, das vor 12 Uhr benutzt wurde. Dies bedeutet, dass das schmutzige Geschirr von jemanden, der um 13 Uhr gekocht hat nicht bis um 01.00 Uhr vom nächsten Tag gespült werden darf. In diesem Fall muss dieses Geschirr bis 23.59 Uhr gespült werden. Geschirrstammplätze sind nicht erlaubt. Dies bedeutet, dass ein Mitbewohner sein Geschirr nicht andauern stehen lassen kann. Die „Parkinggrenzen“ sind 2. 
\subsubsection{Raumgrenzen}
Das dreckige Geschirr darf nur auf die grüne Fläche gestellt werden. Die grüne Fläche wird definiert als die metalische Fläche, die von der Wand, den Brettern und dem Waschbecken begrenzt wird.
\vspace{1cm}

%\includegraphics[scale=0.5]{Waschbeckendraufsicht}

\vspace{1cm}
Das Geschirr darf nur wie folgend auf die grüne Fläche gestellt werden:
\begin{itemize}
\item 3-Teller-Gebot
\item Grüne Fläche Gebot
\vspace{1cm}

%\includegraphics[scale=0.7]{einteller}
%\includegraphics[scale=0.7]{zweiteller}
%\includegraphics[scale=0.7]{dreiteller}
\end{itemize}

Das Geschirr darf wie folgend auf die grüne Fläche nicht gestellt werden

\begin{itemize}
\item Mehr-als-3-Teller-Verbot
\item Flächenüberlappungsverbot
\end{itemize}
%\includegraphics[scale=0.7]{vierteller}
%\includegraphics[scale=0.7]{ueberlappung}

\subsubsection{Geschirrspülungsablauf}

\begin{itemize}
\item Schwamm mit Wasser sättigen\\

%\includegraphics[scale=0.1]{gesaettigterSchwamm}

\item dem gesättigten Schwamm Spülmittel hinzufügen\\
%\includegraphics[scale=0.15]{schwammspuelmittel}

\item Schwamm leicht drücken zum Überprüfen, ob Schaum daraus kommt\\
%\includegraphics[scale=0.15]{swammpressen}

\item Geschirr spülen\\
%\includegraphics[scale=0.25]{geschirrspuelen}

\item Geschirr ausspülen (Bitte beachten: Beim Legen des Geschirrs in den Abtropfgitter muss dieses komplett schaumfrei sein)\\
%\includegraphics[scale=0.15]{geschirrausspuelen}

\item Geschirr abtrocknen\\(Bitte beachten: Beim Legen des Geschirrs in den Abtropfgitter muss dieses komplett schaumfrei sein)\\
%\includegraphics[scale=0.5]{geschirrabtrocknen}

\end{itemize}
\newpage

\centerline{\huge \bf BITTE BEACHTEN!}
\vspace{1cm}
\noindent Hier ist es erlaubt das Geschirr maximal eine Stunde im Abtropfgitter liegen zu lassen unter der Bedingung, das kein Mitbewohner gleich danach spülen muss. Dies muss mit den zu der entsprechenden Zeit anwesenden Mitbewohnern im Küchen- bzw. Wohnzimmerbereich besprochen werden.\\
Das Geschirr muss mit den roten Geschirrtüchern abgetrocknet werden. Die blauen Geschirrtücher sind nur um die eigenen Hände abzutrocknen.\\

%\includegraphics[scale=0.5]{rotestucherlaubt}
%\includegraphics[scale=0.5]{blauestuchnichterlaubt}\\
%\includegraphics[scale=0.5]{blauestucherlaubt}
%\includegraphics[scale=0.5]{blauestuchnichterlaubt}\\
\newpage
\subsection{Mülldienst}
Der Mülldienst erfolgt von demjenigen, an wessen Tür das Mülldienstschild aufhängt.\\
Der Mülldienst erfolgt vom Montag bis Montag. Der Mitbewohner muss in diesem siebentägigen Zeitraum immer die Mülltonnen entleeren, wenn sie voll werden.\\
Am Anfangsmontag bekommt der Mitbewohner des Zimmers „n“ das Mülldienstschild vom Mitbewohner des Zimmers „n-1“ . A315 übergibt das Schild A301.\\
Am Endmontag übergibt der Mitbewohner des Zimmers „n“ das Mülldienstschild dem Mitbewohner des Zimmers „n+1“. Der Mitbewohner, der das Mülldienstschild bekommt überprüft, ob alle Mülltonnen entleert wurden und ob es Biomülltüten gibt und kann sich weigern, das Mülldienst zu erledigen, wenn diese beide Bedingungen nicht eingehalten wurden (siehe Kritikenparagraph).\\
Der Mülldienst schließt die Mülltonnen vom Küchenbereich und Toilettenbereich ein.\\
Wer Mülldienst hat, muss sich darum kümmern, bei Frau Barisch die Biomülltüten zu bekommen.\\
Die nicht Erledigung des Mülldiensts oder dessen Verzögerung stellt ein verstoß gegen den A3-Code dar.

\section{Bad}
\subsection{Duschen}
\begin{itemize}
\item Die Duschen werden nach dem Duschen mit dem Schlauch ausgespült
\item Die Fenster werden nach dem Duschen geöffnet
\end{itemize}

\subsection{Klos}
\begin{itemize}
\item Die Klos werden nach jeder Nutzung mit der Toilettenbürste gereinigt. Man hinterlässt die Klos in besseren Bedingungen, als man sie gefunden hat.
\item Sitzklo: kleine Kabine
\item Stehklo: große Kabine
\item Sich mit Wasser sauber machen: Wenn man aus einer Kultur kommt, in der man sich mit Wasser sauber machen muss, muss man nach jeder Klonutzung beachten, dass der Klobereich nachträglich abgewischt wird.
\item Wenn die Rolle leer ist, die innere braune Rolle in den Müll werfen.
\end{itemize}

 
\subsection{Klopapier} Den Haussprecher rechtzeitig anrufen, wenn es kein Klopapier mehr da ist.

\subsection{Waschmaschine}

\begin{itemize}
\item Beim Einziehen muss eine Kaution in der Höhe von 25€ beim Waschmaschinenminister bezahlt werden. Diese Kaution ist für die Fälle gedacht, in den die Mitbewohner ausziehen ohne seine Schulden beglichen zu haben.
\item Jeder Waschmaschinennutzer muss sich in die dafür vorgesehene Liste eintragen. 
\item Nach dem Ende des Vorganges muss der Waschmaschinennutzer (im folgenden WMN1 gennant) seine Wäsche aus der Waschmaschine tun . Falls WMN1 es nicht tut, so kann der nächste Waschmaschinennutzer (im folgenden WMN2 genannt) bei WMN1 klopfen und ihn darauf hinweisen oder, falls er nicht da ist, die gewaschene Wäsche aus der Waschmaschine herausholen und in ein sauberes Korb legen. Eine Diskussion zwischen WMN1 und WMN2 findet über dieses Thema nicht statt.
\item Die Waschmaschinentür muss grob sauber gemacht und offen gelassen werden.
\item Waschvorgänge zwischen 23.00 Uhr und 08.00 sind untersagt. (Siehe Lautstärke im Gang)
\item Die Waschmaschinenrechnung muss am  Ende jedes Semesters oder 2 Wochen vor dem Ausziehen beglichen werden.
\item Verschüttungen mit Waschmittel oder ähnlichem müssen sofort aufgenommen werden.

\end{itemize}

\subsection{Trockner}

\begin{itemize}
\item Der Türbereich und  der Türfilter müssen mit den Händen gereinigt werden. Der Wasserspeicher muss nach jeder Nutzung des Trockners in den Waschbecken geschüttet werden.
\end{itemize}

\section{Gangbereich}
\subsection{Lautstärke}
 
\begin{itemize}
\item Zwischen 23.00 und 10.00 Uhr vom Folgetag sind Geräusche aller Arten im Flur untersagt (Unterhaltungen, Musik und telefonieren mit offenen Türen, Saugen, usw.).
\item Die Türe werden wie folgt zugemacht: Türklinke nach unten Drucken, Tür langsam und lautlos an den Rahmen anlehnen, Türklinke loslassen, Tür abschließen
\item Die Lautstärke im Zimmer muss die Nachbarn nicht stören. Wenn ja, muss der Mitbewohner leiser werden. (Siehe Kritikenparagraph) 
\end{itemize}

\section{Wohnzimmer}
\subsection{Couchs}
Die Couchs sind um darauf zu sitzen oder zu liegen. Sich auf die Couch hinfallen lassen ist untersagt und ist ein stoß gegen den A3-Code.\\ Sich auf die Couch hinfallen lassen wird wie folgend definiert: Wenn man sein Hintern über eine Höhe von mehr als 10cm auf die Couch hinfallen lässt oder im Allgemeinen wenn man hört, dass sich die Person hingefallen lassen hat. Sich hinsetzen tut man, indem man sein Hintern langsam auf die Couch setzt und danach seinen ganzen Körper entspannt.  \\Jeder Mitbewohner ist verpflichtet, so ein Verhalten anzusprechen und auf das richtige Verhalten hinzuweisen (siehe Kritikenparagraph)


\section{Flurversammlungablauf}
Die ofizielle Sprache der Flurversammlung ist Deutsch. Bei Bedarf kann eine Erklärung auf Englisch erfolgen.\\
Der Flursprecher moderiert die Flurversammlung und erteilt das Rederecht.\\
Die Flurversammlung ist anwesenheitspflicht. Derjenige, der sich nicht mit einem gültigen Grund beim Flursprecher entschuldigt bekommt automatisch das Mülldienstschield und muss diesen erledigen. Das Mülldienstschild muss danach wieder bei der ursprünglich beauftragten Person aufgehängt werden. Alternativ kann der Mitbewohner einen Kuchen für den Flur backen.\\

Die Tagesordnung der Flurversammlung muss wie folgend erfolgen:\\

\begin{enumerate}
\item Begrüßung und Feststellung der Beschlussfähigkeit (Flurschprecher bzw. Flurstellvertreter, Waschmaschinenminister, Kasseminister, Protokollant, Bierminister, 1/2 der ordentlichen Mitbewohnern + 1/2 der restlichen Mitbewohenern) Alle Ämter außer dem Flursprecher dürfen vertreten werden.
\item Ernennung des neuen Flursprechers (Hier stellt sich der alte Flursprecher vor, sagt ob er das Amt behalten möchte und fragt, ob es irgendeinen gibt der gegenkandidieren möchte. Die Flurversammlung stimmt darüber ab, wen sie als Flursprecher haben möchten). Die Flurversammlung wird bis zum Ende vom alten Flursprecher moderiert. Falls ein neuer Flursprecher gewählt wurde, übernimmt er seine Aufgaben unmittelbar nach Beendigung der Flurversammlung.
\item Genehmigung des A3-Codes. Der A3-Code wird per Aklamation genehmigt (Der A3-Code muss in jeder Flurversammlung angepasst werden. Der A3-Code verliert jedoch nicht seine Gültigkeit, wenn es an der Flurversammlung festgestellt wird, der muss aktualisiert werden. Er wird wiederrum sofort aktualisiert und genehmigt). Der A3-Code muss spätestens eine Woche später den Mitbewohnern in Digitalformat zur Verfügung gestellt  und in papierformat auf den Wohnzimmertisch gelegt werden.
\item Festlegung des Termins für die nächste Flurversammlung (dies Erfolgt wie beim Flurputztermin)
\item Festlegung des Termins für den nächsten Flurputz und kontrollieren, ob alle für den Flurputz benötigten Putz- und Desinfektionsmittel vorhanden sind.
\item Festlegung des Termins für die nächste Tour de Chambre.
\item Flurbeitrag: Hier werden vom Kassenminister die Mitbewohnern vorgelesen, die den Flurbeitrag noch nicht bezahlt haben. Die betroffenen Mitbewohner müssen den Beitrag am Ende der Flurversammlung bezahlen oder Auskunft geben, an welchem Datum sie es machen werden.
\item Bericht des Waschmaschinenministers. Benennung des neuen Waschmaschinenministers.
\item Bericht des Bierministers. Benennung des neuen Bierministers.
\item Feedback der Mitbewohner (Hier wird alles gesagt, was schief gelaufen ist, von wem und wie es verbessert werden kann. Alle kritiken müssen konstruktiv sein (siehe definition von konstruktiven Kritiken).
\item Anschaffungen. 

\begin{enumerate}
\item Was fehlt und wie viel es kostet:\\
Hier wird eine Liste gemacht, mit dem was die Mitbewohnern für nötig halten (Es wird in diesem Punkt nicht diskutiert, ob man es braucht oder nicht). Der Mitbewohner, der etwas kaufen möchte muss auch sagen, wie viel es kostet (Er muss sich darum gekümert haben, Produkte zu finden die passend wären). Sonst hat er keinen Anspruch, es anzusprechen.
\item Abstimmung:\\
Es wird über die Anschaffung der einzelnen Produkten abgestimmt.
\item Festlegung des Budgets
\item Festlegung des Mitbewohners, der das Produkt kauft.

\end{enumerate}





\item Feste. Hier wird darüber entschieden, ob man am folgenden Fest teilnehmen möchte.

\begin{enumerate}
\item Abstimmung, ob man überhaupt teilnehmen möchte.
\item Brainstorming über die möglichen Flurstände (Bitte beachten: Hier wird nicht diskutiert ob die Idee machbar ist oder nicht. Es ist nur ein Brainstorming)
\item Wahl der günstigsten Idee per Abstimmung (jeder Mitbewohner darf sich bei der Abstimmung jeder Idee beteiligen. Dies bedeutet, jeder darf bei jeder Idee die Hand hochheben)
\item Wahl der verantwortlichen: 

\begin{enumerate}
\item Ernennung des Standsleiter
\item Aufteilung der Schichten
\item Wer kauft was?
\item Kommunikation mit den Festverantwortlichen
\end{enumerate}

\end{enumerate}


\item Sonstiges (was bisher nicht gesprochen wurde)


\end{enumerate}

\section{Standard A3-Verhalten}
\subsection{Unangebrachtes Verhalten}
Jeder Mitbewohner ist dazu verpflichtet, ein unangebrachtes Verhalten bzw. Eine unangebrachte Aktion sofort anzusprechen und mit den Kritiken von A3-Code zu kritisieren. Unangebrachtes Verhalten kann z.b. Sein:

\begin{itemize}
\item Alles, was gegen den A3-Code verstoßt. 
\item Alles was das normale Wohnen beeinträchtigen kann.
\item Sexistische und Rasistische Kommentare
\item Lichter und Musik anlassen, wenn man weg ist
\item Grobfahrlässiges Verhalten Fluranlagen gegenüber
\item Siehe Beau Mond für weitere Informationen

\end{itemize}



\subsection{Kritiken}
\subsubsection{Konstruktive Kritik}
\begin{itemize}
\item Nicht Schimpfen (Überhaupt nicht schimpfen)
\item Selber die Meinung vertreten 

\begin{itemize}
\item Ich bin der Meinung, dass - OK!
\item Viele Leute sind der Meinung, dass - Falsch!)
\end{itemize}

\item Nicht beleidigen (Keine Spitznamen, nicht Dumm anmachen)
\item Argumente bringen (Ein Argument kann sein „wir machen es alle, also mach das bitte)
\item Verbesserungsvorschlag bringen
\item Face to face kritisieren
\end{itemize}



\subsubsection{Destruktive Kritik} Alle Kritiken, denen ein Element aus den „Konstruktive Kritiken“ fehlt.

\subsubsection{Reagieren auf konstruktive Kritiken}

\begin{itemize}
\item Schweigen und Zuhören, bis die Kritik fertig ist.
\item Bestätigen, dass man die Kritik verstanden hat
\item Kompromiss finden
\item Sagen, ab wann man es besser machen wird (Datum und Uhrzeit)
\end{itemize}

\subsubsection{Reagieren auf destruktive Kritiken}

\begin{itemize}
\item Demjenigen sagen, dass das ein verstoß gegen den A3-Code darstellt und ihn darauf hinweisen, er muss die Kritik neu formulieren oder das Thema unterlassen.
\item Nach Hilfe von anderen Mitbewohnern suchen, wenn er es nicht macht.
\end{itemize}

\subsubsection{Mitbewohner weigert sich, eine konstruktive Kritik anzunehmen}
Unterstützung von anderen Mitbewohnern suchen, sodass diese Person einsieht, dass sie einen Fehler gemacht hat.

\end{document}